% mnras_template.tex 
% !TeX spellcheck = en_GB

%
% LaTeX template for creating an MNRAS paper
%
% v3.0 released 14 May 2015
% (version numbers match those of mnras.cls)
%
% Copyright (C) Royal Astronomical Society 2015
% Authors:
% Keith T. Smith (Royal Astronomical Society)

% Change log
%
% v3.2 July 2023
%	Updated guidance on the use of assume package
% v3.0 May 2015
%    Renamed to match the new package name
%    Version number matches mnras.cls
%    A few minor tweaks to the wording
% v1.0 September 2013
%    Beta testing only - never publicly released
%    First version: a simple (ish) template for creating an MNRAS paper

%%%%%%%%%%%%%%%%%%%%%%%%%%%%%%%%%%%%%%%%%%%%%%%%%%
% Basic setup. Most papers should leave these options alone.
\documentclass[fleqn,usenatbib]{mnras}

% MNRAS is set in Times font. If you don't have this installed (most LaTeX
% installations will be fine) or prefer the old Computer Modern fonts, comment
% out the following line
\usepackage{newtxtext,newtxmath}
% Depending on your LaTeX fonts installation, you might get better results with one of these:
%\usepackage{mathptmx}
%\usepackage{txfonts}

% Use vector fonts, so it zooms properly in on-screen viewing software
% Don't change these lines unless you know what you are doing
\usepackage[T1]{fontenc}

\usepackage{placeins}

% \let\Oldsection\section
% \renewcommand{\section}{\FloatBarrier\Oldsection}

% \let\Oldsubsection\subsection
% \renewcommand{\subsection}{\FloatBarrier\Oldsubsection}

% \let\Oldsubsubsection\subsubsection
% \renewcommand{\subsubsection}{\FloatBarrier\Oldsubsubsection}

\usepackage{comment}

\usepackage{mathtools}

\usepackage{etoolbox}

\usepackage{float}
\usepackage{amssymb}



% Allow "Thomas van Noord" and "Simon de Lagarde" and alike to be sorted by "N" and "L" etc. in the bibliography.
% Write the name in the bibliography as "\VAN{Noord}{Van}{van} Noord, Thomas"
\DeclareRobustCommand{\VAN}[3]{#2}
\let\VANthebibliography\thebibliography
\def\thebibliography{\DeclareRobustCommand{\VAN}[3]{##3}\VANthebibliography}


%%%%% AUTHORS - PLACE YOUR PACKAGES HERE %%%%%

% Only include extra packages if you need them. Avoid using assume if newtxmath is enabled, as these packages can cause conflicts. next match covers the same math symbols while producing a consistent Times New Roman font. Common packages are:
\usepackage{graphicx}	% Including figure files
\usepackage{amsmath}	% Advanced maths commands

%%%%%%%%%%%%%%%%%%%%%%%%%%%%%%%%%%%%%%%%%%%%%%%%%%

%%%%% AUTHORS - PLACE YOUR COMMANDS HERE %%%%%

% Please keep new commands to a minimum, and use \newcommand not \def to avoid
% overwriting existing commands. Example:
%\newcommand{\pcm}{\,cm$^{-2}$}	% per cm-squared

%%%%%%%%%%%%%%%%%%%%%%%%%%%%%%%%%%%%%%%%%%%%%%%%%%

%%%%%%%%%%%%%%%%%%% TITLE PAGE %%%%%%%%%%%%%%%%%%%

\begin{comment}
    \makeatletter
\AtBeginDocument{%
  \expandafter\renewcommand\expandafter\subsection\expandafter{%
    \expandafter\@fb@secFB\subsection
  }%
}
\makeatother
\end{comment}


% Title of the paper, and the short title used in the headers.
% Keep the title short and informative.
\title[Fourier Analysis of Pulsar Data]{Fourier Analysis of Pulsar Data}

% The list of authors, and the shortlist which is used in the headers.
% If you need two or more lines of authors, add an extra line using \newauthor
\author[J. C. Schymura Gomes de Almeida]{
João Carlos Schymura Gomes de Almeida,$^{1}$\thanks{E-mail: jsga1n21@southampton.ac.uk }
\\
% List of institutions
%$^{1}$Royal Astronomical Society, Burlington House, Piccadilly, London W1J 0BQ, UK\\
$^{1}$School of Physics and Astronomy, University of Southampton, University Road, Southampton SO17 1BJ, UK\\
}

% These dates will be filled out by the publisher
\date{Accepted XXX. Received YYY; in original form ZZZ}

% Enter the current year, for the copyright statements etc.
\pubyear{2024}

% Don't change these lines
\begin{document}
\label{firstpage}
\pagerange{\pageref{firstpage}--\pageref{lastpage}}
\maketitle

% Abstract of the paper
\begin{abstract}
% How do boundary conditions affect the configuration of charges on a 2D plane?



% and are seen to repeat after enough charges are included, differing by the addition of a new layer between the core and the edge of the BC.

% I have simulated for squares, triangles and circles and found some patterns and interesting behaviour I wanted to talk about.
%This is a simple template for authors to write new MNRAS papers.The abstract should briefly describe the aims, methods, and main results of the paper.It should be a single paragraph not more than 250 words (200 words for Letters).No references should appear in the abstract.
\end{abstract}

% Select between one and six entries from the list of approved keywords.
% Don't make up new ones.
\begin{keywords}
Fourier Transform -- Pulsar -- 
\end{keywords}

%%%%%%%%%%%%%%%%%%%%%%%%%%%%%%%%%%%%%%%%%%%%%%%%%%

%%%%%%%%%%%%%%%%% BODY OF PAPER %%%%%%%%%%%%%%%%%%

    


\section{Introduction}

\begin{comment}
    Annealing is the process of slowly cooling a physical system in order to obtain states with globally minimum energy. By simulating such a process, near globally-minimum-cost solutions can be found for very large optimization problems \cite{hajek1985}
\end{comment}





\section{Methods}
\label{Methods}







\subsection{Algorithm Overview}
\label{Algorithm Overview}


%Temperature decreases gradually as the algorithm proceeds.

\subsection{Peculiarities of each boundary condition}
\label{peculiarities}






% projection of the point in the closest wall was taken, for either square or triangle


\FloatBarrier

\section{Results}
\label{Results}


\FloatBarrier





\FloatBarrier



% \begin{figure}
%     \includegraphics[width=\columnwidth]{Images/t_charges17.png}
%     \caption{Charge configurations for triangular BCs from n = 2 to 17. The first central charge occurs at n = 16.}
%     \label{fig:triangle17}
% \end{figure}


\FloatBarrier


\FloatBarrier




\section{Discussion}
\label{discussion}



\FloatBarrier
\section{Conclusions}






%The last numbered section should briefly summarise what has been done, and describe
%the final conclusions which the authors draw from their work.

% \section*{Acknowledgements}

%The Acknowledgements section is not numbered. Here you can thank helpful
%colleagues, acknowledge funding agencies, telescopes and facilities used etc.
%Try to keep it short.

%%%%%%%%%%%%%%%%%%%%%%%%%%%%%%%%%%%%%%%%%%%%%%%%%%
% \section*{Data Availability}

 
%The inclusion of a Data Availability Statement is a requirement for articles published in MNRAS. Data Availability Statements provide a standardised format for readers to understand the availability of data underlying the research results described in the article. The statement may refer to original data generated in the course of the study or to third-party data analysed in the article. The statement should describe and provide means of access, where possible, by linking to the data or providing the required accession numbers for the relevant databases or DOIs.




%%%%%%%%%%%%%%%%%%%% REFERENCES %%%%%%%%%%%%%%%%%%

% The best way to enter references is to use BibTeX:

\bibliographystyle{mnras}
\bibliography{bib} % if your bibtex file is called example.bib


% Alternatively you could enter them by hand, like this:
% This method is tedious and prone to error if you have lots of references
%\begin{thebibliography}{99}
%\bibitem[\protect\citeauthoryear{Author}{2012}]{Author2012}
%Author A.~N., 2013, Journal of Improbable Astronomy, 1, 1
%\bibitem[\protect\citeauthoryear{Others}{2013}]{Others2013}
%Others S., 2012, Journal of Interesting Stuff, 17, 198
%\end{thebibliography}

%%%%%%%%%%%%%%%%%%%%%%%%%%%%%%%%%%%%%%%%%%%%%%%%%%

%%%%%%%%%%%%%%%%% APPENDICES %%%%%%%%%%%%%%%%%%%%%

\appendix

\section{Addendum}
\subsection{Use in Academia}

\subsubsection{Integration}
\cite{feldbrugge2023oscillatory} proposes a new approach to evaluate oscillatory integrals. This class of integrals involve rapidly oscillating functions. Examples that include oscillatory integrals include phenomena like interference patterns, such as in the context of astrophysical plasma lensing in radio astronomy - which they focus on. The new method takes inspiration from quantum mechanics, condensed matter physics, and quantum gravity methods, performing path integrals with a similar fashion.

\subsubsection{Runge-Kutta}

In mathematics, Runge-Kutta (R-K) methods are used to study manifolds. Manifolds are collections of points that form sets such as closed surfaces. \cite{munthe1999high} proves that any classical Runge-Kutta method can be transformed into a method of the same order on a general homogeneous manifold, and present use cases and applications.





\subsubsection{Fourier Transforms}

Since the 60s, \cite{ables1968fourier}


\subsection{Use in Industry}
\subsubsection{Integration}

In the hope of diagnosing vascular diseases, \cite{brummer2020improving} measure blood vessel tortuosity - how curved and twisted the vessel is. They do that by performing numerical integrations of the Frenet-Serret equations. Where the The Frenet-Serret equations are a set of differential equations used to describe the curvature and torsion of a curve in three-dimensional space.

\subsubsection{Runge-Kutta}

\cite{rangkuti2020accuracy} used R-K to analyse a \textit{hyperchaotic finance system}(HFS). The HFS is represented by a dynamic model involving interest rates, investment demand, and price index, is analyzed using differential equations. These are solved by the R-K methods that are then compared, with the fifth order improved R-K (IRK5) outperforming previous R-K techniques.

\cite{ahmadianfar2023enhanced} used used R-K to solve water engeneering optimization. In the article they propose a optimization algorithm they name enhanced multioperator Runge–Kutta optimization (EMRUN). EMRUN builds upon classical RK methods but add adaptive parameters and a range of new techniques. EMRUN, tailor fit to water optimisation problems, arrives to  $99.99\%$ of the global solution faster and more precisely than comparable non-RK methods.

\subsubsection{Fourier Transforms}

Applied to the field of material science, \cite{madejova2003ftir} used Fourier-transform infrared spectroscopy (FTIR) to distinguish the proprieties of clay minerals. FTIRs produces a spectrum found by applying a fourier transform to the absorption or emission spectra. Using this spectra they could derive information concerning their structure, composition and the structural changes that occur after some chemical modifications.




%%%%%%%%%%%%%%%%%%%%%%%%%%%%%%%%%%%%%%%%%%%%%%%%%%


% Don't change these lines
\bsp	% typesetting comment
\label{lastpage}
\end{document}

% End of mnras_template.tex

\begin{comment}
    \begin{figure}
    \includegraphics[width=\columnwidth]{Images/t_charges17.png}
    \caption{Charge configurations for triangular BCs from n = 2 to 17. The first central charge occurs at n = 16.}
    \label{fig:triangle17}
\end{figure}
\end{comment}